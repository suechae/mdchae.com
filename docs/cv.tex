\documentclass[12pt,]{article}
\usepackage[sc, osf]{mathpazo}
\usepackage{amssymb,amsmath}
\usepackage{ifxetex,ifluatex}
\usepackage{fixltx2e} % provides \textsubscript
\ifnum 0\ifxetex 1\fi\ifluatex 1\fi=0 % if pdftex
  \usepackage[T1]{fontenc}
  \usepackage[utf8]{inputenc}
\else % if luatex or xelatex
  \ifxetex
    \usepackage{mathspec}
  \else
    \usepackage{fontspec}
  \fi
  \defaultfontfeatures{Ligatures=TeX,Scale=MatchLowercase}
\fi
% use upquote if available, for straight quotes in verbatim environments
\IfFileExists{upquote.sty}{\usepackage{upquote}}{}
% use microtype if available
\IfFileExists{microtype.sty}{%
\usepackage{microtype}
\UseMicrotypeSet[protrusion]{basicmath} % disable protrusion for tt fonts
}{}
\usepackage[margin=1in]{geometry}




\setlength{\emergencystretch}{3em}  % prevent overfull lines
\providecommand{\tightlist}{%
  \setlength{\itemsep}{0pt}\setlength{\parskip}{0pt}}
\setcounter{secnumdepth}{0}
% Redefines (sub)paragraphs to behave more like sections
\ifx\paragraph\undefined\else
\let\oldparagraph\paragraph
\renewcommand{\paragraph}[1]{\oldparagraph{#1}\mbox{}}
\fi
\ifx\subparagraph\undefined\else
\let\oldsubparagraph\subparagraph
\renewcommand{\subparagraph}[1]{\oldsubparagraph{#1}\mbox{}}
\fi
\usepackage{kotex}
\setmainhangulfont[ItalicFont={*},ItalicFeatures={FakeSlant=.167}]{NanumMyeongjo}

% Now begins the stuff that I added.
% ----------------------------------

% Custom section fonts
\usepackage{sectsty}
\sectionfont{\rmfamily\mdseries\large\bf}
\subsectionfont{\rmfamily\mdseries\normalsize\itshape}



% Make parskips rather than indent with lists.
\usepackage{parskip}
\usepackage{titlesec}
\titlespacing\section{0pt}{12pt plus 4pt minus 2pt}{4pt plus 2pt minus 2pt}
\titlespacing\subsection{0pt}{12pt plus 4pt minus 2pt}{4pt plus 2pt minus 2pt}

% Use fontawesome. Note: you'll need TeXLive 2015. Update.


% Fancyhdr, as I tend to do with these personal documents.
\usepackage{fancyhdr,lastpage}
\pagestyle{fancy}
\renewcommand{\headrulewidth}{0.0pt}
\renewcommand{\footrulewidth}{0.0pt}
\lhead{}
\chead{}
\rhead{}
\lfoot{
\cfoot{\scriptsize  채의수, M.D. - Curriculum Vitae }}
\rfoot{\scriptsize \thepage/{\hypersetup{linkcolor=black}\pageref{LastPage}}}

% Always load hyperref last.
\usepackage{hyperref}
\PassOptionsToPackage{usenames,dvipsnames}{color} % color is loaded by hyperref

\hypersetup{unicode=true,
            pdftitle={채의수, M.D.:  Curriculum Vitae (Curriculum Vitae)},
            pdfauthor={채의수, M.D.},
            pdfkeywords={curriculum vitae, Rmarkdown, CV},
            colorlinks=true,
            linkcolor=Black,
            citecolor=Blue,
            urlcolor=Black,
            breaklinks=true, bookmarks=true}
\urlstyle{same}  % don't use monospace font for urls

\begin{document}


\centerline{\huge \bf 채의수, M.D.}

\vspace{2 mm}

\hrule

\vspace{2 mm}

\moveleft.5\hoffset\centerline{산부인과 전문의 (Board-Certified OB/GYN)}

\moveleft.5\hoffset\centerline{ \emph{Email:} \href{mailto:}{\href{mailto:suechae@gmail.com}{\nolinkurl{suechae@gmail.com}}} \hspace{1 mm} \emph{Phone:}  +82-10-3324-0522  \hspace{1 mm}       | \emph{Updated:} \today}

\vspace{2 mm}

\hrule


\section{학력}

\textbf{석사} \hfill 2007년 3월 - 2011년 2월\\
\emph{부산대학교 의학전문대학원 졸업 \textbar{} 의학 전공}
\hfill 부산광역시

\textbf{학사} \hfill 2001년 9월 - 2005년 5월\\
\emph{미시간주립대학교 졸업 \textbar{} Pre-med (Human Biology) 전공}
\hfill 미시간주, 미국

\section{경력}

\textbf{원장} \hfill 2017년 6월 - 2018년 5월 (현재)\\
\emph{애플산부인과} \hfill 경기도 성남시

\textbf{레지던트} \hfill 2013년 9월 - 2017년 5월\\
\emph{서울아산병원 산부인과} \hfill 서울특별시

\textbf{인턴} \hfill 2012년 9월 - 2013년 8월\\
\emph{가톨릭중앙의료원} \hfill 서울특별시

\textbf{Postdoc Fellow} \hfill 2011년 3월 - 2012년 8월\\
\emph{National Cancer Institute, 미국국립보건원} \hfill 메릴랜드주, 미국

\section{면허}

\textbf{산부인과 전문의 (번호: 7340)} \hfill 2018년 3월

\textbf{의사면허 (번호: 106498)} \hfill 2011년 2월

\hypertarget{-}{%
\section{소속 학회}\label{-}}

\emph{대한산부인과학회} \hfill 2013년 - 현재

\emph{대한생식의학회} \hfill 2013년 - 현재

\emph{대한모체태아의학회} \hfill 2013년 - 현재

\emph{대한산부인과초음파학회} \hfill 2013년 - 현재

\pagebreak

\section{수상}

\textbf{전공의 우수 연구 표창장} \hfill 2017년 2월\\
\emph{서울아산병원장}

\textbf{실습우수상} \hfill 2010년 6월\\
\emph{양산부산대학교병원 소아청소년과장}

\textbf{서브인턴 장학금} \hfill 2009년 7월\\
\emph{부산대학교 의학전문대학원장}

\textbf{호주 Flinders 의대 연수 장학금} \hfill 2008년 7월\\
\emph{부산대학교 의학전문대학원장}

\section{\texorpdfstring{\href{https://scholar.google.co.kr/citations?user=J2bg_TAAAAAJ\&hl=en}{논문}}{논문}}

Chae, U, M. Lee, H. Kim, H. Won, K. Kim, H. Goo, J. Ko and J. Park
(2018). ``Prenatal diagnosis of isolated coronary arteriovenous fistula:
a first case report in Korea''. In: \emph{Obstetrics and Gynecology
Science} 61.1, pp.~161-164.

Chae, U, J. Y. Min, S. H. Kim, H. J. Ihm, Y. S. Oh, S. Y. Park, H. D.
Chae, C. Kim and B. M. Kang (2016). ``Decreased Progesterone Receptor
B/A Ratio in Endometrial Cells by Tumor Necrosis Factor-Alpha and
Peritoneal Fluid from Patients with Endometriosis''. In: \emph{Yonsei
medical journal} 57.6, pp.~1468-1474.

Arnold, M. A, C. A. Arnold, G. Li, U. Chae, R. El-Etriby, C. R. Lee and
M. Tsokos (2013). ``A unique pattern of INI1 immunohistochemistry
distinguishes synovial sarcoma from its histologic mimics''. In:
\emph{Human pathology} 44.5, pp.~881-887.

Jeong, D. W, J. G. Lee, S. Lee, Y. J. Kim, J. H. Bae, D. H. Kim, Y. H.
Yi, Y. H. Cho and U. Chae (2012). ``Potential effect of insulin
resistance and cardiovascular risk factors on metabolic syndrome in
subjects with normal fasting plasma glucose levels''. In:
\emph{International Journal of Diabetes in Developing Countries} 32.2,
pp.~75-81.

\hypertarget{-}{%
\section{자격 사항}\label{-}}

\textbf{임상연구개론: KGCP 포함} \hfill 2017년 5월\\
\emph{질병관리본부}

\textbf{심화심폐소생술 ACLS Provider} \hfill 2015년 5월\\
\emph{미국심장학회}

\section{외국어}

\textbf{영어 - 상급} \emph{외국인환자 진료 가능, 영문 논문 및 전문적
글작성 가능}

\pagebreak

\section{추천인}

\textbf{김종혁, M.D.,Ph.D.}\\
서울아산병원 기획실장, 울산대학교 의과대학 산부인과교실 교수

\textbf{노환중, M.D.,Ph.D.}\\
양산부산대학교 병원장, 부산대학교 의학전문대학원 이비인후과교실 교수

\textbf{\href{https://connects.catalyst.harvard.edu/Profiles/display/Person/124975}{Maria
Tsokos, M.D.}}\\
하버드의대 해부병리학교실 교수

\end{document}
